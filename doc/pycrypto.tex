\documentclass[letterpaper,10pt]{article}
\usepackage{fullpage}


%opening
\title{A Working Introduction to Crypto with PyCrypto}
\author{Kyle Isom\\coder@kyleisom.net\\https://www.github.com/kisom/crypto\_intro}


\begin{document}

\maketitle

\section{Introduction}
Recently at work I have been using the
PyCrypto\footnote{https://www.dlitz.net/software/pycrypto/} libraries quite
a bit. The documentation is pretty good, but there are a few areas
that took me a bit to figure out. In this post, I’ll be writing up
a quick overview of the PyCrypto library and cover some general
things to know when writing cryptographic code in general. I’ll go
over symmetric, public-key, hybrid, and message authentication
codes. Keep in mind this is a quick introduction and a lot of gross
simplifications are made. For a more complete introduction to
cryptography, take a look at the references at the end of this
article. This article is just an appetite-whetter - if you have a
real need for information security you should hire an expert. Real
data security goes beyond this quick introduction (you wouldn’t
trust the design and engineering of a bridge to a student with a
quick introduction to civil engineering, would you?)

Some quick terminology: for those unfamiliar, I introduce the following
terms:

\begin{itemize}

  \item plaintext: the original message

  \item ciphertext: the message after cryptographic transformations
  are applied to obscure the original message.

  \item encrypt: producing ciphertext by applying cryptographic
  transformations to plaintext.

  \item decrypt: producing plaintext by applying cryptographic
  transformations to ciphertext.

  \item cipher: a particular set of cryptographic transformations
  providing means of both encryption and decryption.

  \item hash: a set of cryptographic transformations that take a
  large input and transform it to a unique (typically fixed-size)
  output.  For hashes to be cryptographically secure, collisions
  should be practically nonexistent. It should be practically
  impossible to determine the input from the output.

\end{itemize}

Cryptography is an often misunderstood component of information
security, so an overview of what it is and what role it plays is in
order. There are four major roles that cryptography plays:

\begin{itemize}

  \item confidentiality: ensuring that only the intended recipients
  receive the plaintext of the message.

  \item data integrity: the plaintext message arrives unaltered.

  \item entity authentication: the identity of the sender is verified.
  An entity may be a person or a machine.

  \item message authentication: the message is verified as having
  been unaltered.

\end{itemize}

Note that cryptography is used to obscure the contents of a message and
verify its contents and source. It will \emph{not} hide the fact that two
entities are communicating.

There are two basic types of ciphers: symmetric and public-key ciphers.
A symmetric key cipher employs the use of shared secret keys. They also
tend to be much faster than public-key ciphers. A public-key cipher is
so-called because each key consists of a private key which is used to
generate a public key. Like their names imply, the private key is kept
secret while the public key is passed around. First, I’ll take a look at
a specific type of symmetric ciphers: block ciphers.


\section{Block Ciphers}
There are two further types of symmetric keys: stream and block
ciphers.  Stream ciphers operate on data streams, i.e. one byte at
a time. Block ciphers operate on blocks of data, typically 16 bytes
at a time. The most common block cipher and the standard one you
should use unless you have a very good reason to use another one
is the
AES\footnote{https://secure.wikimedia.org/wikipedia/en/wiki/Advanced\_Encryption\_Standard}
block cipher, also documented in FIPS PUB
197\footnote{http://csrc.nist.gov/publications/fips/fips197/fips-197.pdf}.
AES is a specific subset of the Rijndael cipher. AES uses block
size of 128-bits (16 bytes); data should be padded out to fit the
block size - the length of the data block must be multiple of the
block size. For example, given an input of \verb|ABCDABCDABCDABCD ABCDABCDABCDABCD|
no padding would need to be done.  However, given
\verb|ABCDABCDABCDABCD ABCDABCDABCD| an additional 4 bytes of padding
would need to be added. A common padding scheme is to use \verb|0x80|
as the first byte of padding, with \verb|0x00| bytes filling out the
rest of the padding.  With padding, the previous example would look
like: \verb|ABCDABCDABCDABCD| \verb|ABCDABCDABCD\x80\x00\x00\x00|.

Writing a padding function is pretty easy:

\begin{verbatim}
def pad_data(data):
    # return data if no padding is required
    if len(data) % 16 == 0: 
        return data

    # subtract one byte that should be the 0x80
    # if 0 bytes of padding are required, it means only
    # a single \x80 is required.

    padding_required     = 15 - (len(data) % 16)

    data = '%s\x80' % data
    data = '%s%s' % (data, '\x00' * padding_required)

    return data
\end{verbatim}

Similarly, removing padding is also easy:

\begin{verbatim}
def unpad_data(data):
    if not data: 
        return data

    data = data.rstrip('\x00')
    if data[-1] == '\x80':
        return data[:-1]
    else:
        return data
\end{verbatim}

Encryption with a block cipher requires selecting a block
mode\footnote{https://en.wikipedia.org/wiki/Block\_cipher\_mode}.  By far the
most common mode used is \emph{cipher block chaining} or \emph{CBC} mode.
Other modes include \emph{counter (CTR)}, \emph{cipher feedback (CFB)}, and
the extremely insecure \emph{electronic codebook (ECB)}. CBC mode is the
standard and is well-vetted, so I will stick to that in this tutorial.
Cipher block chaining works by XORing the previous block of ciphertext
with the current block. You might recognise that the first block
has nothing to be XOR'd with; enter the \emph{initialisation
vector}\footnote{https://en.wikipedia.org/wiki/Initialization\_vector}.  This
comprises a number of randomly-generated bytes of data the same
size as the cipher's block size. This initialisation vector should
random enough that it cannot be recovered.

One of the most critical components to encryption is properly
generating random data. Fortunately, most of this is handled by the
PyCrypto library’s \verb|Crypto.Random.OSRNG| module. You should know
that the more entropy sources that are available (such as network
traffic and disk activity), the faster the system can generate
cryptographically-secure random data. I've written a function that
can generate a
\emph{nonce}\footnote{https://secure.wikimedia.org/wikipedia/en/wiki/Cryptographic\_nonce}
suitable for use as an initialisation vector. This will work on a
UNIX machine; the comments note how easy it is to adapt it to a
Windows machine. This function requires a version of PyCrypto at
least 2.1.0 or higher.

\begin{verbatim}
import Crypto.Random.OSRNG.posix as RNG

def generate_nonce():
    return RNG.new().read(AES.block_size)
\end{verbatim}

I will note here that the python `random` module is completely
unsuitable for cryptography (as it is completely deterministic). You
shouldn’t use it for cryptographic code.

Symmetric ciphers are so-named because the key is shared across any
entities.  There are three key sizes for AES: 128-bit, 192-bit, and
256-bit, aka 16-byte, 24-byte, and 32-byte key sizes. Instead, we just
need to generate 32 random bytes (and make sure we keep track of it)
and use that as the key:

\begin{verbatim}
KEYSIZE = 32


def generate_key():
    return RNG.new().read(KEY_SIZE)
\end{verbatim}

We can use this key to encrypt and decrypt data. To encrypt, we
need the initialisation vector (i.e. a nonce), the key, and the
data. However, the IV isn't a secret. When we encrypt, we'll prepend
the IV to our encrypted data and make that part of the output. We
can (and should) generate a completely random IV for each new
message.

\begin{verbatim}
def encrypt(data, key):
    """
    Encrypt data using AES in CBC mode. The IV is prepended to the
    ciphertext.
    """
    data = pad_data(data)
    ivec = generate_nonce()
    aes = AES.new(key, AES.MODE_CBC, ivec)
    ctxt = aes.encrypt(data)
    return ivec + ctxt


def decrypt(ciphertext, key):
    """
    Decrypt a ciphertext encrypted with AES in CBC mode; assumes the IV
    has been prepended to the ciphertext.
    """
    if len(ciphertext) <= AES.block_size:
        raise Exception("Invalid ciphertext.")
    ivec = ciphertext[:AES.block_size]
    ciphertext = ciphertext[AES.block_size:]
    aes = AES.new(key, AES.MODE_CBC, ivec)
    data = aes.decrypt(ciphertext)
    return unpad_data(data)
\end{verbatim}

However, this is only part of the equation for securing messages:
AES only gives us confidentiality. Remember how we had a few other
criteria? We still need to add integrity and authenticity to our
process. Readers with some experience might immediately think of
hashing algorithms, like MD5 (which should be avoided like the
plague) and SHA. The problem with these is that they are malleable:
it is easy to change a digest produced by one of these algorithms,
and there is no indication it's been changed. We need, a hash
function that uses a key to generate the digest; the one we'll use
is called HMAC. We do not want the same key used to encrypt the
message; we should have a new, freshly generated key that is the
same size as the digest's output size (although in many cases, this
will be overkill).

In order to encrypt properly, then, we need to modify our code a bit.
The first thing, you need to know is that HMAC is based on a
particular SHA function. Since we're using AES-256, we'll use SHA-384.
We say our message tags are computed using HMAC-SHA-384. This
produces a 48-byte digest. Let's add a few new constants in, and
update the \verb|KEYSIZE| variable:

\begin{verbatim}
__aes_keylen = 32
__tag_keylen = 48
KEYSIZE = __aes_keylen + __tag_keylen
\end{verbatim}

Now, let's add message tagging in:

\begin{verbatim}
import Crypto.Hash.HMAC as HMAC
import Crypto.Hash.SHA384 as SHA384

def new_tag(ciphertext, key):
    """Compute a new message tag using HMAC-SHA-384."""
    return HMAC.new(key, msg=ciphertext, digestmod=SHA384).digest()
\end{verbatim}

Here's our updated encrypt function:

\begin{verbatim}
def encrypt(data, key):
    """
    Encrypt data using AES in CBC mode. The IV is prepended to the
    ciphertext.
    """
    data = pad_data(data)
    ivec = generate_nonce()
    aes = AES.new(key[:__aes_keylen], AES.MODE_CBC, ivec)
    ctxt = aes.encrypt(data)
    tag = new_tag(ctxt, key[__aes_keylen:]) 
    return ivec + ctxt + tag
\end{verbatim}

Decryption has a snag: what we want to do is check to see if the
message tag matches what we think it should be. However, the Python
\verb|==| operator stops matching on the first character it finds that
doesn't match. This opens a verification based on the \verb|==| operator
to a timing attack. Without going into much detail, note that several
cryptosystems have fallen prey to this exact attack; the keyczar
system, for example, use the \verb|==| operator and suffered an attack
on the system. We'll use the \verb|streql| package (i.e.
\verb|pip install streql|) to perform a constant-time comparison
of the tags.

\begin{verbatim}
import streql


__taglen = 48


def verify_tag(ciphertext, key):
    """Verify the tag on a ciphertext."""
    tag_start = len(ciphertext) - __taglen
    data = ciphertext[:tag_start]
    tag = ciphertext[tag_start:]
    actual_tag = new_tag(data, key)
    return streql.equals(actual_tag, tag)
\end{verbatim}

We'll also change our decrypt function to return a tuple: the
original message (or \verb|None| on failure), and a boolean that will be
True if the tag was authenticated and the message decrypted

\begin{verbatim}
def decrypt(ciphertext, key):
    """
    Decrypt a ciphertext encrypted with AES in CBC mode; assumes the IV
    has been prepended to the ciphertext.
    """
    if len(ciphertext) <= AES.block_size:
        return None, False
    tag_start = len(ciphertext) - __TAG_LEN
    ivec = ciphertext[:AES.block_size]
    data = ciphertext[AES.block_size:tag_start]
    if not verify_tag(ciphertext, key[__AES_KEYLEN:]):
        return None, False
    aes = AES.new(key[:__AES_KEYLEN], AES.MODE_CBC, ivec)
    data = aes.decrypt(data)
    return unpad_data(data), True
\end{verbatim}

We could also generate a key using a passphrase, but this is
significantly more complex: you should use a key derivation algorithm,
such as PBKDF2\footnote{https://en.wikipedia.org/wiki/Pbkdf2}. A function
to derivate a key from a passphrase will also need to store the
salt that goes with the passphrase. PBKDF2 will generate a salt to
go along with the password; the salt is analogous to the initialisation
vector and can be stored alongside the password.

That should cover the basics of block cipher encryption. We’ve
gone over key generation, padding, and encryption / decryption.


\section{ASCII-Armouring}
I'm going to take a quick detour and talk about ASCII armouring. If
you've played with the crypto functions above, you'll notice they
produce an annoying dump of binary data that can be a hassle to
deal with. One common technique for making the data a little bit
easier to deal with is to encode it with base64. There are a
few ways to incorporate this into python:

\subsection{Absolute Base64 Encoding}
The easiest way is to just base64 encode everything in the encrypt
function. Everything that goes into the decrypt function should be
in base64 - if it's not, the \verb|base64| decoding will throw an error:
you could catch this and then try to decode it as binary data.

\subsection{A Simple Header}
A slightly more complex option, and the one I adopt in this article,
is to use a \verb|\x00| as the first byte of the ciphertext for
binary data, and to use \verb|\x41| (an ASCII `A') for ASCII encoded
data. This will increase the complexity of the encryption and
decryption functions slightly. My modified functions look like this
now:

\begin{verbatim}
def encrypt(data, key, armour=False):
    """
    Encrypt data using AES in CBC mode. The IV is prepended to the
    ciphertext.
    """
    data = pad_data(data)
    ivec = generate_nonce()
    aes = AES.new(key[:__AES_KEYLEN], AES.MODE_CBC, ivec)
    ctxt = aes.encrypt(data)
    tag = new_tag(ivec+ctxt, key[__AES_KEYLEN:])
    if armour:
        return '\x41' + (ivec + ctxt + tag).encode('base64')
    else:
        return '\x00' + ivec + ctxt + tag

def decrypt(ciphertext, key):
    """
    Decrypt a ciphertext encrypted with AES in CBC mode; assumes the IV
    has been prepended to the ciphertext.
    """
    if ciphertext[0] == '\x41':
        ciphertext = ciphertext[1:].decode('base64')
    else:
        ciphertext = ciphertext[1:]
    if len(ciphertext) <= AES.block_size:
        return None, False
    tag_start = len(ciphertext) - __TAG_LEN
    ivec = ciphertext[:AES.block_size]
    data = ciphertext[AES.block_size:tag_start]
    if not verify_tag(ciphertext, key[__AES_KEYLEN:]):
        return None, False
    aes = AES.new(key[:__AES_KEYLEN], AES.MODE_CBC, ivec)
    data = aes.decrypt(data)
    return unpad_data(data), True
\end{verbatim}

\subsection{A More Complex Container}
There are more complex ways to do it (and you’ll see it with the
public keys in the next section) that involve putting the base64
into a container of sorts that contains additional information about
the key.


\section{Public Key Cryptography}
The original version of this document had examples of using RSA
cryptography with Python. However, RSA should be avoided for modern
secure systems, and I haven't been using Python, so I'm not very
familiar with the options for elliptic curve cryptography. Rather
than encouraging the use of a weaker cipher, I've opted to elide
this. A possible starting point is to look at Yann Guibet's
\verb|pyelliptic|\footnote{https://github.com/yann2192/pyelliptic} package.
It should provide ECDSA for signatures, and ECDH for encryption.

\end{document}
